\RequirePackage{amsmath}
%\RequirePackage[l2tabu, orthodox]{nag}
\documentclass[french, english]{llncs}
\input{preamble/packages}
\input{preamble/math_basics}
\input{preamble/math_mine}
\input{preamble/redac}
\input{preamble/draw}
\input{preamble/acronyms}

\begin{document}
\title{Non linear preference models: why, how?}
\author{Olivier Cailloux\inst{1} \and Sébastien Destercke\inst{2}}
\institute{
LAMSADE\\
\email{olivier.cailloux@dauphine.fr}
\and HDS
}
\hypersetup{
	pdfsubject={preference modeling},
	pdfkeywords={keyw1, keyw2}
}
\maketitle

\abstract{Literature involving preferences of artificial agents or human beings often assume their preferences are linear, that is, the preference can be represented using a complete transitive antisymmetric binary relation. Much has been written however on more complex, or more interesting, models of preferences. In this article we review some of the reasons that have been put forward to justify more complex modeling, and review some of the techniques that have been proposed to obtain models of such preferences. (Optional: we connect to various related literature about argumentation, …)}

\section{Introduction}
Here are possible subjects we might want to talk about.
\begin{itemize}
	\item reasons for rebel preferences: non transitivity; may be linear according to some other description (Sen)
	\item reasons for modeling rebel preferences (VS simpler approx): use the resulting undeterminacy (lability)
	\item descriptive VS prescriptive; preferences are complete or transitive
	\item why ranking ≠ choice ≠ sorting
	\item maximizing does not require transitivity
	\item nec and possible
	\item class of models by generality
	\item axiomatics
	\item computational difficulties
	\item argu : expliquer le modèle ; éliciter le modèle ; choix classe de fonctions
\end{itemize}

Here is a possible approach we might want to use.

\begin{description}
	\item Models of preferences can have (at least) two interpretations: normative or descriptive. Goals and (sometimes) tools to model preferences and reasonable hypothesis that can be postulated about the shape of the preferences vary according to these two lines.
	\item (Reasonable) hypothesis about descriptive preferences. Why it’s more difficult than can be thought naïvely. 
	\item Reasonable hypothesis about normative preferences. Why it’s more difficult than can be thought naïvely. Talk about incompleteness. Applying normative approaches: talk about prescription a la Roy…
	\item Present usual models (Savage? MAVT?) and alternative ones (belief functions? Electre?)
	\item Links to argumentation theory: build argumentative models that explain / justify a decision.
	\item Path to validation of normative models. Talk about empirical social choice; reflective equilibrium.
\end{description}
\end{document}
