\RequirePackage{amsmath}
%\RequirePackage[l2tabu, orthodox]{nag}
\documentclass[french, english]{llncs}
\input{preamble/packages}
\input{preamble/math_basics}
\input{preamble/math_mine}
\input{preamble/redac}
\input{preamble/draw}
\input{preamble/acronyms}

\begin{document}
\title{Non linear preference models: why, how?}
\author{Olivier Cailloux\inst{1} \and Sébastien Destercke\inst{2}}
\institute{
LAMSADE\\
\email{olivier.cailloux@dauphine.fr}
\and HDS
}
\hypersetup{
	pdfsubject={preference modeling},
	pdfkeywords={keyw1, keyw2}
}
\maketitle

\abstract{Literature involving preferences of artificial agents or human beings often assume their preferences are linear, that is, the preference can be represented using a complete transitive antisymmetric binary relation. Much has been written however on more complex, or more interesting, models of preferences. In this article we review some of the reasons that have been put forward to justify more complex modeling, and review some of the techniques that have been proposed to obtain models of such preferences. (Optional: we connect to various related literature about argumentation, …)}

\section{Introduction}
Here is a possible approach we might want to use.

\begin{itemize}
	\item What’s a model of preference? Example with weighted sum. But: forbids some possibly interesting evaluations; need to recode evaluations and hence is dependent of irrelevant alternatives; w1 > w2 unclear.
	\item Utility: is holistic. Forbids incomparability. Mandates transitivity.
	\item Models of preferences can have (at least) two interpretations: normative or descriptive. Goals and (sometimes) tools to model preferences and reasonable hypothesis that can be postulated about the shape of the preferences vary according to these two lines.
	\item (Reasonable) hypothesis about descriptive preferences. Why it’s more difficult than can be thought naïvely. 
	\item Reasonable hypothesis about normative preferences. Why it’s more difficult than can be thought naïvely. Talk about incompleteness. Applying normative approaches: talk about prescription a la Roy…
	\item Focus on incompleteness rather than non-transitivity. Can be lack of information or intrinsic.
	\item Present usual models (Savage? MAVT?) and alternative ones (belief functions? Electre?)?
	\item Talk about choice under uncertainty wrt incompleteness.
	\item Impact of relaxations on problems of choice, ranking, classification.
	\item How to obtain information under incompleteness hypothesis, meaning and interpretation.
	\item (If space permits) Links to argumentation theory: build argumentative models that explain / justify a decision.?
	\item (If space permits) Path to validation of normative models. Talk about empirical social choice; reflective equilibrium. Use the resulting indeterminacy (lability). Maximizing without transitivity or completeness?
\end{itemize}

\end{document}
